\documentclass[aspectratio=1610, xcolor=dvipsnames]{beamer}
\usetheme{Pittsburgh}

\usepackage{amsmath}
\usepackage{amsfonts}
\usepackage{amssymb}
\usepackage{mathtools}
\usepackage{siunitx}
\usepackage[version=4]{mhchem}

\usepackage[T1]{fontenc}

\usepackage{animate}
\usepackage{biblatex}

\usepackage{booktabs}
\usepackage{cmbright}
\usepackage[gen]{eurosym}
\usepackage{fancyvrb}
\usepackage{multicol}
\usepackage{setspace}
\usepackage[absolute,overlay]{textpos}

\usepackage{xcolor}
\usepackage{graphicx}

\newcommand{\COO}{\ensuremath{\mathrm{CO_2}} }
%\usepackage{natbib}
\usepackage{cellspace}
\setlength\cellspacetoplimit{4pt}
\setlength\cellspacebottomlimit{4pt}

% plotting with tikz and pgfplots
\usepackage{pgfplots}
\usepackage{tikz}
\usepackage{hyperref}
\usetikzlibrary{shapes.geometric, arrows, positioning, fit, shapes.arrows}
\usetikzlibrary{decorations.pathreplacing, decorations.markings}
\pgfplotsset{compat = 1.16}
% make tikz compatible with \againframe
\tikzset{onslide/.code args={<#1>#2}{%
    \only<#1>{\pgfkeysalso{#2}} % \pgfkeysalso doesn't change the path
}}
\tikzset{alt/.code args={<#1>#2#3}{%
    \alt<#1>{\pgfkeysalso{#2}}{\pgfkeysalso{#3}} % \pgfkeysalso doesn't change the path
}}
\tikzset{temporal/.code args={<#1>#2#3#4}{%
    \temporal<#1>{\pgfkeysalso{#2}}{\pgfkeysalso{#3}}{\pgfkeysalso{#4}} % \pgfkeysalso doesn't change the path
}}

% horizontal alignment of \underbraces
\newcommand{\tempstrut}{\rule[-1.2\baselineskip]{0pt}{1.5\baselineskip}}

%gets rid of bottom navigation bars
\setbeamertemplate{footline}[frame number]{}

%gets rid of bottom navigation symbols
\setbeamertemplate{navigation symbols}{}

% colors
\definecolor{bokugreen}{rgb}{0.302, 0.675, 0.149}
\setbeamercolor{title}{fg=bokugreen, bg=white}
\setbeamercolor{frametitle}{fg=bokugreen}
\setbeamercolor{itemize item}{fg=bokugreen}
\setbeamercolor{enumerate item}{fg=bokugreen}
\setbeamercolor{local structure}{fg=bokugreen}

% wider slide
\newcommand\Wider[2][3em]{%
    \makebox[\linewidth][c]{%
        \begin{minipage}{\dimexpr\textwidth+#1\relax}
            \raggedright#2
        \end{minipage}%
    }%
}

% title page
\setbeamertemplate{title page}[default][colsep=-4bp,rounded=true]
\title[The Cost of Undisturbed Landscapes]{The Cost of Undisturbed Landscapes}
\author{Sebastian Wehrle}
\institute{Institute for Sustainable Economic Development \\University of Natural Resources and Life Sciences, Vienna}
\date{January 12, 2021}
% section page
\setbeamertemplate{section page}{
    \begin{textblock*}{4cm}
        (11.5cm, 4cm) % {block width} (coords)
            {\LARGE \insertsection}
    \end{textblock*}
}
% subsection page
\setbeamertemplate{subsection page}{
    \begin{textblock*}{4cm}
        (11.5cm, 4cm) % {block width} (coords)
            {\LARGE \insertsection} \\ \bigskip \insertsubsection
    \end{textblock*}
}

\begin{document}
{
    \begin{frame}[noframenumbering,plain]
        \maketitle
        \centering
        \includegraphics[height=1.7cm]{./figures/boku-logo}\\
    \end{frame}
}

    \begin{frame}
        \frametitle{Preface}
        \framesubtitle{Motivation for research}
        \structure{Wind power potentials -- state of the art in a nutshell} \\~\\
        %\medskip
        %\structure{How much wind power \emph{can} be installed?} \\~\\
        \begin{itemize}
            \item Wind power potentials typically derived by step-wise exclusion of areas
            \begin{itemize}
                \item technical restrictions (e.g.\ elevation)
                \item legal restrictions (e.g.\ protected areas)
                \item resource restrictions (i.e.\ sufficient wind at location)
            \end{itemize} %\\~\\
            \item Assumption that wind power is feasible in all areas not excluded
        \end{itemize}
        \medskip
        Yet, increasing understanding that "social acceptance" matters
        \medskip
        \begin{itemize}
            \item Exclude areas where erection of wind turbines is "socially unacceptable"
            \item In the literature, such areas are identified, for example, based on
            \begin{itemize}
                \item participatory processes
                \item approximation of public authorities' licensing decisions through "scenicness"
            \end{itemize} %\\~\\
            %\item Binary attributions
            %\item No guidance on choice of method
            %\item How is selection of areas justified?
        \end{itemize}
    \end{frame}

    \begin{frame}
        \frametitle{Preface}
        \framesubtitle{Motivation for research}
        \structure{Wind power potentials -- my take on advancing the state-of-the-art} \\~\\
        \begin{itemize}
            \item Abandon binary attributions which fall short of complexity involved
            \item Realize that there is no guaranteed way to come up with unanimous ("socially accepted") decisions in a society
            \begin{itemize}
                \item How much wind power \emph{can} be deployed ultimately is a matter of power
            \end{itemize}
        \end{itemize}
        \medskip
        Instead, consider how much wind power \emph{should} be installed
        \begin{itemize}
            \item Minimize damage caused
            \item Weigh cost and benefit we are willing to take (as a society)
        \end{itemize}
    \end{frame}

    \begin{frame}
        \frametitle{How much wind power \emph{should} be installed}
        \begin{columns}[T]
            \begin{column}{7cm}
                \begin{block}{The cost of undisturbed landscapes}
                    \begin{itemize}
                        \item change in electricity generation cost
                        \item change in GHG emissions
                        \item change in local air pollution
                        \item change in grid connection cost
                        \item change in capital construction cost
                        \item change in O\&M cost
                        \item change in cost of supply security
                    \end{itemize}
                \end{block}
            \end{column}

            \begin{column}{7cm}
                \begin{block}{The benefit of undisturbed landscapes}
                    \begin{itemize}
                        \item avoided impact on ecosystem services
                        \item avoided impact on amenity value of landscape
                        \item avoided impacts on wildlife
                        \item avoided loss of property value
                    \end{itemize}
                \end{block}
            \end{column}
        \end{columns}
    \end{frame}

    \begin{frame}
        \frametitle{The cost of undisturbed landscapes}
        \framesubtitle{Some background}
        \begin{itemize}
            \item By 2030, \SI{100}{\percent}\footnote[frame]{System services and industry own consumption,
            accounting for $\sim$\SI{10}{\percent} of total electricity demand and $\sim$\SI{40}{\percent} of \ce{CO2} emissions from power
            generation are exempt} of Austria's electricity demand to be sourced from renewables
            \item Sufficient potentials for large-scale expansion only for wind and solar power
        \end{itemize}
        \medskip
        {%
            \setbeamercolor{block body}{bg=bokugreen!30, fg=black}
            \setbeamertemplate{blocks}[rounded][shadow=false]
            \begin{block}{}
                \begin{itemize}
                    \item Not disturbing landscapes by wind turbines means substituting wind turbines with PV modules (potentially open-space)
                \end{itemize}
            \end{block}
        }%
        \medskip
        \begin{itemize}
            \item Substituting wind turbines with PV modules changes system operation and investment
            \item Which cost arises due to this substitution?
            \item In our case, this is the \emph{opportunity cost of wind power}, as the cost-minimizing system relies
            (almost) entirely on wind power
        \end{itemize}
    \end{frame}

    \begin{frame}
        \frametitle{The cost of undisturbed landscapes}
        \framesubtitle{Methodology}
        %, which includes all operation and
        %investment cost of the (domestic) electricity system, the cost of emissions (we also include air pollutants), plus
        %the cost (revenue) from electricity imports (exports).

        \medskip
        {%
            \setbeamercolor{block title}{bg=bokugreen, fg=white}
            \setbeamercolor{block body}{bg=bokugreen!20, fg=black}
            \setbeamertemplate{blocks}[rounded][shadow=false]
            \begin{block}{Approximation of the opportunity cost of wind turbines $OC_w$}
                %To approximate the opportunity cost of wind turbines $OC_w$, we proceed as follows:
                \begin{enumerate}
                    \item determine unconstrained level of RET that leads to minimal system cost $c_{net}$ \label{step1}
                    \item restrict the admissible expansion of the least-cost RET by a small margin, $\Delta w$, and observe
                    resulting change in system cost $\Delta c_{net}$ \label{step2}
                    \item approximate the opportunity cost of wind power by \label{step3}
                    \begin{align*}
                        OC_w = \frac{\Delta c_{net}}{\Delta w}
                    \end{align*}
                    \item repeat steps \ref{step2} and \ref{step3} till no more wind turbines can be erected
                \end{enumerate}
            \end{block}
        }%
    \end{frame}

    \begin{frame}
        \frametitle{The cost of undisturbed landscapes}
        \framesubtitle{Assumptions}
        We quantify the opportunity cost of wind power with the numerical power system model \emph{medea}, which we set up
        to resemble the Austrian and German electricity and district heating systems in 2030.
        \begin{columns}[T]
        {%
            \setbeamercolor{block title}{bg=bokugreen, fg=white}
            \setbeamercolor{block body}{bg=bokugreen!20, fg=black}
            \setbeamertemplate{blocks}[rounded][shadow=false]
            \begin{column}{7cm}
                \begin{block}{Model assumptions}
                    \begin{itemize}
                        \item perfect foresight, partial-equilibrium model of electricity and district heating sectors in
                        AT and DE
                        \item technology-rich representation of supply to meet price-inelastic demand
                        \item economic dispatch in hourly resolution plus investment
                    \end{itemize}
                \end{block}
            \end{column}

            \begin{column}{7cm}
                \begin{block}{Scenario assumptions (2030)}
                    \begin{itemize}
                        \item DE: nuclear and (partial) coal exit
                        \item DE: RET expansion as in EEG 2017
                        \item AT: must generate 79.3 TWh/a from RES
                        \begin{itemize}
                            \item + 5 TWh/a hydropower
                            \item + 1 TWh/a electricity from biomass
                        \end{itemize}
                        \item weather as in 2016 (a low wind year)
                    \end{itemize}
                \end{block}
            \end{column}
        }%
        \end{columns}
        \bigskip
        \url{https://github.com/inwe-boku/medea}
    \end{frame}

    \begin{frame}
        \frametitle{Change in annual net system cost -- Austria, 2030}
        \begin{figure}
            \includegraphics[width=0.95\textwidth]{./figures/cost_components_base_25.pdf} \label{fig:cost_breakdown}
        \end{figure}
    \end{frame}

    \begin{frame}
        \frametitle{Monthly electricity generation by source -- Austria, 2030}
        \framesubtitle{(a) maximum wind power, (b) maximum solar power}
        \begin{figure}
            \includegraphics[width=0.85\textwidth]{./figures/dispatch_AT.pdf} \label{fig:monthly_gen}
        \end{figure}
    \end{frame}

    \begin{frame}
        \frametitle{Annual opportunity cost of wind power -- Austria, 2030}
        \framesubtitle{PV capital cost: \euro 625/kWp}
        \begin{figure}
            \includegraphics[width=0.95\textwidth]{./figures/oc_base.pdf} \label{fig:opportunity_cost_base}
        \end{figure}
    \end{frame}

    \begin{frame}
        \frametitle{Sensitivity of annual opportunity cost of wind power -- Austria, 2030}
        \begin{figure}
            \includegraphics[width=0.95\textwidth]{./figures/Sensitivities.pdf} \label{fig:sensitivities}
        \end{figure}
    \end{frame}

\end{document}