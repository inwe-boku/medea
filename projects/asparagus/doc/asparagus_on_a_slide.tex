\documentclass[aspectratio=1610, xcolor=dvipsnames]{beamer}
\usetheme{Pittsburgh}

\usepackage{amsmath}
\usepackage{amsfonts}
\usepackage{amssymb}
\usepackage{mathtools}
\usepackage{siunitx}
\usepackage[version=4]{mhchem}
\usepackage{qrcode}

\usepackage[T1]{fontenc}

\usepackage{animate}
\usepackage{biblatex}

\usepackage{booktabs}
\usepackage{cmbright}
\usepackage[gen]{eurosym}
\usepackage{fancyvrb}
\usepackage{multicol}
\usepackage{setspace}
\usepackage[absolute,overlay]{textpos}

\usepackage{xcolor}
\usepackage{graphicx}

\newcommand{\COO}{\ensuremath{\mathrm{CO_2}} }
%\usepackage{natbib}
\usepackage{cellspace}
\setlength\cellspacetoplimit{4pt}
\setlength\cellspacebottomlimit{4pt}

% plotting with tikz and pgfplots
\usepackage{pgfplots}
\usepackage{tikz}
\usepackage{hyperref}
\usetikzlibrary{shapes.geometric, arrows, positioning, fit, shapes.arrows}
\usetikzlibrary{decorations.pathreplacing, decorations.markings}
\pgfplotsset{compat = 1.16}
% make tikz compatible with \againframe
\tikzset{onslide/.code args={<#1>#2}{%
    \only<#1>{\pgfkeysalso{#2}} % \pgfkeysalso doesn't change the path
}}
\tikzset{alt/.code args={<#1>#2#3}{%
    \alt<#1>{\pgfkeysalso{#2}}{\pgfkeysalso{#3}} % \pgfkeysalso doesn't change the path
}}
\tikzset{temporal/.code args={<#1>#2#3#4}{%
    \temporal<#1>{\pgfkeysalso{#2}}{\pgfkeysalso{#3}}{\pgfkeysalso{#4}} % \pgfkeysalso doesn't change the path
}}

% horizontal alignment of \underbraces
\newcommand{\tempstrut}{\rule[-1.2\baselineskip]{0pt}{1.5\baselineskip}}

%gets rid of bottom navigation bars
\setbeamertemplate{footline}[frame number]{}

%gets rid of bottom navigation symbols
\setbeamertemplate{navigation symbols}{}

% colors
\definecolor{bokugreen}{rgb}{0.302, 0.675, 0.149}
\setbeamercolor{title}{fg=bokugreen, bg=white}
\setbeamercolor{frametitle}{fg=bokugreen}
\setbeamercolor{itemize item}{fg=bokugreen}
\setbeamercolor{enumerate item}{fg=bokugreen}
\setbeamercolor{local structure}{fg=bokugreen}

% wider slide
\newcommand\Wider[2][3em]{%
    \makebox[\linewidth][c]{%
        \begin{minipage}{\dimexpr\textwidth+#1\relax}
            \raggedright#2
        \end{minipage}%
    }%
}

% title page
\setbeamertemplate{title page}[default][colsep=-4bp,rounded=true]
\title[The Cost of Undisturbed Landscapes]{The Cost of Undisturbed Landscapes}
\author{Sebastian Wehrle}
\institute{Institute for Sustainable Economic Development \\University of Natural Resources and Life Sciences, Vienna}
\date{January 12, 2021}
% section page
\setbeamertemplate{section page}{
    \begin{textblock*}{4cm}
        (11.5cm, 4cm) % {block width} (coords)
            {\LARGE \insertsection}
    \end{textblock*}
}
% subsection page
\setbeamertemplate{subsection page}{
    \begin{textblock*}{4cm}
        (11.5cm, 4cm) % {block width} (coords)
            {\LARGE \insertsection} \\ \bigskip \insertsubsection
    \end{textblock*}
}

\begin{document}
{
    \begin{frame}[noframenumbering,plain]
        \maketitle
        \centering
        \includegraphics[height=1.7cm]{./figures/boku-logo}\\
    \end{frame}
}

    \begin{frame}
        \frametitle{Preface}
        \framesubtitle{Motivation for research}
        \structure{Wind power potentials -- state of the art in a nutshell} \\~\\
        %\medskip
        %\structure{How much wind power \emph{can} be installed?} \\~\\
        \begin{itemize}
            \item Wind power potentials typically derived by step-wise exclusion of areas
            \begin{itemize}
                \item technical restrictions (e.g.\ elevation)
                \item legal restrictions (e.g.\ protected areas)
                \item resource restrictions (i.e.\ sufficient wind at location)
            \end{itemize} %\\~\\
            \item Assumption that wind power is feasible in all areas not excluded
        \end{itemize}
        \medskip
        Yet, increasing understanding that "social acceptance" matters
        \medskip
        \begin{itemize}
            \item Exclude areas where erection of wind turbines is "socially unacceptable"
            \item In the literature, such areas are identified, for example, based on
            \begin{itemize}
                \item participatory processes
                \item approximation of public authorities' licensing decisions through "scenicness"
            \end{itemize} %\\~\\
            %\item Binary attributions
            %\item No guidance on choice of method
            %\item How is selection of areas justified?
        \end{itemize}
    \end{frame}

    \begin{frame}
        \frametitle{Preface}
        \framesubtitle{Motivation for research}
        \structure{Wind power potentials -- my take on advancing the state-of-the-art} \\~\\
        \begin{itemize}
            \item Abandon binary attributions which fall short of complexity involved
            \item Realize that there is no guaranteed way to come up with unanimous ("socially accepted") decisions in a society
            \begin{itemize}
                \item Determining potentials means asking how much wind power \emph{can} be deployed
                \item Yet, this ultimately is a matter of power
            \end{itemize}
        \end{itemize}
        \medskip
        Instead, consider how much wind power \emph{should} be installed
        \begin{itemize}
            \item Minimize damage caused
            \item Weigh cost and benefit we are willing to take (as a society)
        \end{itemize}
    \end{frame}

    \begin{frame}
        \frametitle{How much wind power \emph{should} be installed}
        \begin{columns}[T]
            \begin{column}{7cm}
                \begin{block}{The cost of undisturbed landscapes}
                    \begin{itemize}
                        \item change in electricity generation cost
                        \item change in GHG emissions
                        \item change in local air pollution
                        \item change in grid connection cost
                        \item change in capital construction cost
                        \item change in O\&M cost
                        \item change in cost of supply security
                    \end{itemize}
                \end{block}
            \end{column}

            \begin{column}{7cm}
                \begin{block}{The benefit of undisturbed landscapes}
                    \begin{itemize}
                        \item avoided impact on ecosystem services
                        \item avoided impact on amenity value of landscape
                        \item avoided impacts on wildlife
                        \item avoided loss of property value
                    \end{itemize}
                \end{block}
            \end{column}
        \end{columns}
    \end{frame}

    \begin{frame}
        \frametitle{The cost of undisturbed landscapes}
        \framesubtitle{Introduction}
        \begin{itemize}
            \item By 2030, \SI{100}{\percent}\footnote[frame]{System services and industry own consumption,
                accounting for $\sim$\SI{10}{\percent} of total electricity demand and $\sim$\SI{40}{\percent} of \ce{CO2} emissions from power
                generation are exempt} of Austria's electricity demand to be sourced from renewables
            \item Sufficient potentials for large-scale expansion only for wind and solar power
        \end{itemize}
        \medskip
        {%
            \setbeamercolor{block body}{bg=structure.fg!20, fg=black}
            \setbeamertemplate{blocks}[rounded][shadow=false]
            \begin{block}{}
                \begin{itemize}
                    \item Not disturbing landscapes by wind turbines means substituting wind turbines with PV modules (potentially open-space)
                \end{itemize}
            \end{block}
        }%
        \medskip
        \begin{itemize}
            \item Substituting wind turbines with PV modules changes system operation and investment
            \item Which cost arises due to this substitution?
            \item In our case, this is the \emph{opportunity cost of wind power}, as the cost-minimizing system relies
            (almost) entirely on wind power
        \end{itemize}
    \end{frame}

    \begin{frame}
        \frametitle{The cost of undisturbed landscapes}
        \framesubtitle{Policy consequences}
        Immediate corollary of the "100\%"-policy:
        \begin{itemize}
            \item  Austria turns from a net importer of electricity (2016: 7.2 TWh, 9.9\% of total electricity
            consumption) into a net exporter
            \begin{itemize}
                \item Heat generation requires running (fossil) co-generation units
                \item (Fossil) generation also required to secure supply at all times
            \end{itemize}
            \item This is not the case under a (equivalent) carbon pricing regime
        \end{itemize}
    \end{frame}

    \begin{frame}
        \frametitle{The cost of undisturbed landscapes}
        \framesubtitle{Methodology}
        %, which includes all operation and
        %investment cost of the (domestic) electricity system, the cost of emissions (we also include air pollutants), plus
        %the cost (revenue) from electricity imports (exports).

        \medskip
        {%
            \setbeamercolor{block title}{bg=structure.fg!0, fg=structure.fg}
            \setbeamercolor{block body}{bg=structure.fg!0, fg=black}%{bg=bokugreen!20, fg=black}
            \setbeamertemplate{blocks}[rounded][shadow=false]
            \begin{block}{Approximation of the opportunity cost of wind turbines $OC_w$}
                %To approximate the opportunity cost of wind turbines $OC_w$, we proceed as follows:
                \medskip
                \begin{enumerate}
                    \item determine unconstrained level of RET that leads to minimal system cost $c_{net}$ \label{step1}
                    \item restrict the admissible expansion of the least-cost RET by a small margin, $\Delta w$, and observe
                    resulting change in system cost $\Delta c_{net}$ \label{step2}
                    \item approximate the opportunity cost of wind power $OC_w$ by \label{step3}
                    \begin{align*}
                        OC_w = \frac{\Delta c_{net}}{\Delta w}
                    \end{align*}
                    \item repeat steps~\ref{step2} and~\ref{step3} till no more wind turbines can be erected
                \end{enumerate}
            \end{block}
        }%
    \end{frame}

    \begin{frame}
        \frametitle{The cost of undisturbed landscapes}
        \framesubtitle{Assumptions}
        {%
            \setbeamercolor{block body}{bg=structure.fg!20, fg=black}
            \setbeamertemplate{blocks}[rounded][shadow=false]
            \begin{block}{}
                We quantify the opportunity cost of wind power with the numerical power system model \emph{medea}, which we set up
                to resemble the Austrian and German electricity and district heating systems in 2030
            \end{block}
        }%
        \begin{columns}[T]
            %{%
            %\setbeamercolor{block title}{bg=bokugreen, fg=white}
            %\setbeamercolor{block body}{bg=bokugreen!20, fg=black}
            %\setbeamertemplate{blocks}[rounded][shadow=false]
            \begin{column}{7cm}
                %\begin{block}{Model assumptions}
                \structure{Model assumptions}
                \begin{itemize}
                    \item perfect foresight, partial-equilibrium model of electricity and district heating sectors in
                    AT and DE
                    \item technology-rich representation of supply to meet price-inelastic demand
                    \item economic dispatch in hourly resolution plus investment
                \end{itemize}
                %\end{block}
            \end{column}

            \begin{column}{7cm}
                %\begin{block}{Scenario assumptions (2030)}
                \structure{Scenario assumptions (2030)}
                \begin{itemize}
                    \item DE: nuclear and (partial) coal exit
                    \item DE: RET expansion as in EEG 2017
                    \item AT: at least 74.8 TWh/a from RES
                    \begin{itemize}
                        \item + 5 TWh/a hydropower
                        \item + 1 TWh/a electricity from biomass
                    \end{itemize}
                    \item weather as in 2016 (a low wind year)
                \end{itemize}
                %\end{block}
            \end{column}
            %}%
        \end{columns}
        \bigskip
        \begin{center}
            \url{https://github.com/inwe-boku/medea}
        \end{center}
    \end{frame}

    \begin{frame}
        \frametitle{Results}
        \begin{columns}[T]
            \begin{column}{10.6 cm}
                \begin{figure}
                    \caption{Change in annual net system cost -- Austria, 2030}
                    \includegraphics[width=1.0\textwidth]{./figures/cost_components_base_25.pdf} \label{fig:cost_breakdown}
                \end{figure}
            \end{column}
            \begin{column}{5.0 cm}
                \bigskip
                \begin{itemize}
                    %\item "100\%"-policy necessarily turns Austria from a net importer (2016: 7.2 TWh) into a
                    %net exporter of electricity as fossil generation needed (heat, system services)
                    \item each avoided GW wind power replaced by 1.98 GWp PV
                    \item mixture of RET slightly reduce variance and increases minimal renewable generation by a small margin
                    \item resource adequacy drives O\&M cost
                    \item lower exports also mean less fossil generation
                \end{itemize}
            \end{column}
        \end{columns}
    \end{frame}

    \begin{frame}
        \frametitle{Results}
        \begin{columns}[T]
            \begin{column}{10.6 cm}
                \begin{figure}
                    \caption{\parbox[t]{8.6cm}{Monthly electricity generation by source -- Austria, 2030 (a) maximum wind power, (b) maximum solar power}}
                    \includegraphics[width=0.875\textwidth]{./figures/dispatch_AT.pdf} \label{fig:monthly_gen}
                \end{figure}
            \end{column}
            \begin{column}{5.0 cm}
                \bigskip
                \begin{itemize}
                    \item Wind power is favourably complementary to run-of-river generation
                    \item Solar PV positively correlated with run-of-river generation
                    \item In PV-based system more imports needed in winter, while export price in summer lower
                    \item PV pushes $CO_2$ emissions from Austria to Germany
                \end{itemize}
            \end{column}
        \end{columns}
    \end{frame}

    \begin{frame}
        \frametitle{Results}
        %\framesubtitle{PV capital cost: \euro 625/kWp}
        \begin{columns}[T]
            \begin{column}{10.6 cm}
                \begin{figure}
                    \caption{\parbox[t]{8.4cm}{Annual opportunity cost of wind power -- Austria, 2030 PV capital cost: \euro 625/kWp}}
                    \includegraphics[width=0.975\textwidth]{./figures/oc_base.pdf} \label{fig:opportunity_cost_base}
                \end{figure}
            \end{column}
            \begin{column}{5.0 cm}
                \bigskip
                \begin{itemize}
                    \item Figure shows marginal system cost of not erecting wind turbines
                    \item Policy target implies marginal opportunity cost of \euro 22\,500 per MW and year
                    \item Equals \euro 1.3 mn over lifetime of a single 3.5 MW wind turbine (discounted at 5\%)
                \end{itemize}
            \end{column}
        \end{columns}
    \end{frame}

%    \begin{frame}
%        \frametitle{Summary}
%        \begin{itemize}
%            \item So-called 100\% goal necessarily turns Austria into a net exporter of electricity
%            \item Choice of RET impacts installed capacities, system operation, and, most importantly, trade flows
%            \item Austria's wind resources are favourable for exporting to Germany
%            \item Export revenues accrue to electricity producers, not consumers
%            \item Consumers benefit from low electricity prices
%            \item Compensation for neighbors of wind parks makes sense
%        \end{itemize}
%    \end{frame}

    \begin{frame}
        \frametitle{Discussion}
        \begin{itemize}
            \item Renewable resource quality held constant as capacity is expanded
            \begin{itemize}
                \item Hölting et a. (2016) estimate 45 TWh of wind resources available in Austria at equal or lower cost
                than we assumed
                \item We consider additions of at most 32 TWh (at a carbon price of \euro 100 per t)
            \end{itemize}
            \item Sub-national electricity grids neglected
            \begin{itemize}
                \item grid cost of open-space PV and wind power should be similar
                \item cost of rooftop PV in distribution grid underresearched
                \item some estimates imply distribution cost not substantially different from wind power in transmission grids
                \item however, huge variability of estimates
            \end{itemize}
            \item External cost of open-space PV assumed to be zero
            \begin{itemize}
                \item external cost of open-space PV underresearched
                \item mixed findings from working papers ranging from "no impact" to "small impact with limited significance"
            \end{itemize}
        \end{itemize}
    \end{frame}

    \begin{frame}
        \frametitle{Conclusions}
        \begin{itemize}
            \item If cost of $CO_2$ low and PV realized as open-space, utility-scale installations,
            wind power does not come at a system cost advantage
            \item With high $CO_2$ cost, a preference for rooftop PV, or external costs of
            open-space, utility-scale PV, wind power can reduce total system cost (including external cost)
            \item Wind power increases electricity's current account surplus, which accrues to exporters.
            Compensation of affected residents could improve distributional fairness and, potentially, increase
            acceptance of wind power \\
            \medskip
            \item Trade effects lead to much higher system cost impacts of capacity decisions than currently acknowledged in literature
            \item In contrast to (relative) market value, opportunity cost of renewables consistent with fully renewable energy systems
        \end{itemize}
    \end{frame}

    \begin{frame}
        \frametitle{Thank you!}
        %\vspace{1 cm}
        \begin{center}
            \structure{%
                \href{https://arxiv.org/abs/2006.08009}{The Cost of Undisturbed Landscapes}
            }\\%
            \vspace{0.6cm}
            \qrcode[height=2cm]{https://arxiv.org/abs/2006.08009}\\
            \vspace{0.6 cm}
            \href{https://arxiv.org/abs/2006.08009}{https://arxiv.org/abs/2006.08009}
        \end{center}
    \end{frame}

    \begin{frame}
        \frametitle{Results}

        \begin{columns}[T]
            \begin{column}{10.6 cm}
                \begin{figure}
                    \caption{\parbox[t]{8.0cm}{Sensitivity of annual opportunity cost of wind power (Cubic fit across $CO_2$ price scenarios)}}
                    \includegraphics[width=0.975\textwidth]{./figures/Sensitivities.pdf} \label{fig:sensitivities}
                \end{figure}
            \end{column}
            \begin{column}{5.4 cm}
                \bigskip
                \begin{itemize}
                    \item \emph{Low capital cost of PV} assumes overnight cost of PV of \euro 560/kWp
                    \item \emph{No bottleneck} assumes 10 GW transmission capacity
                \end{itemize}
            \end{column}
        \end{columns}
    \end{frame}

    \begin{frame}
        \frametitle{Results}
        \begin{columns}[T]
            \begin{column}{10.6 cm}
                \begin{figure}
                    \caption{\parbox[t]{8.0cm}{Sensitivity of optimal wind power expansion}}
                    \includegraphics[width=0.975\textwidth]{./figures/S_PVCost_P_WindOnshore.pdf} \label{fig:sens_wind_cap}
                \end{figure}
            \end{column}
            \begin{column}{5.0 cm}
                \bigskip
                \begin{itemize}
                    \item If RET expansion incentivised by carbon price (not "100\%"-policy), optimal wind power expansion is
                    robust against capital cost of PV
                \end{itemize}
            \end{column}
        \end{columns}
    \end{frame}

\end{document}